\documentclass[]{article}
\usepackage{lmodern}
\usepackage{amssymb,amsmath}
\usepackage{ifxetex,ifluatex}
\usepackage{fixltx2e} % provides \textsubscript
\ifnum 0\ifxetex 1\fi\ifluatex 1\fi=0 % if pdftex
  \usepackage[T1]{fontenc}
  \usepackage[utf8]{inputenc}
\else % if luatex or xelatex
  \ifxetex
    \usepackage{mathspec}
  \else
    \usepackage{fontspec}
  \fi
  \defaultfontfeatures{Ligatures=TeX,Scale=MatchLowercase}
\fi
% use upquote if available, for straight quotes in verbatim environments
\IfFileExists{upquote.sty}{\usepackage{upquote}}{}
% use microtype if available
\IfFileExists{microtype.sty}{%
\usepackage{microtype}
\UseMicrotypeSet[protrusion]{basicmath} % disable protrusion for tt fonts
}{}
\usepackage[margin=1in]{geometry}
\usepackage{hyperref}
\hypersetup{unicode=true,
            pdfborder={0 0 0},
            breaklinks=true}
\urlstyle{same}  % don't use monospace font for urls
\usepackage{longtable,booktabs}
\usepackage{graphicx,grffile}
\makeatletter
\def\maxwidth{\ifdim\Gin@nat@width>\linewidth\linewidth\else\Gin@nat@width\fi}
\def\maxheight{\ifdim\Gin@nat@height>\textheight\textheight\else\Gin@nat@height\fi}
\makeatother
% Scale images if necessary, so that they will not overflow the page
% margins by default, and it is still possible to overwrite the defaults
% using explicit options in \includegraphics[width, height, ...]{}
\setkeys{Gin}{width=\maxwidth,height=\maxheight,keepaspectratio}
\IfFileExists{parskip.sty}{%
\usepackage{parskip}
}{% else
\setlength{\parindent}{0pt}
\setlength{\parskip}{6pt plus 2pt minus 1pt}
}
\setlength{\emergencystretch}{3em}  % prevent overfull lines
\providecommand{\tightlist}{%
  \setlength{\itemsep}{0pt}\setlength{\parskip}{0pt}}
\setcounter{secnumdepth}{5}
% Redefines (sub)paragraphs to behave more like sections
\ifx\paragraph\undefined\else
\let\oldparagraph\paragraph
\renewcommand{\paragraph}[1]{\oldparagraph{#1}\mbox{}}
\fi
\ifx\subparagraph\undefined\else
\let\oldsubparagraph\subparagraph
\renewcommand{\subparagraph}[1]{\oldsubparagraph{#1}\mbox{}}
\fi

%%% Use protect on footnotes to avoid problems with footnotes in titles
\let\rmarkdownfootnote\footnote%
\def\footnote{\protect\rmarkdownfootnote}

%%% Change title format to be more compact
\usepackage{titling}

% Create subtitle command for use in maketitle
\newcommand{\subtitle}[1]{
  \posttitle{
    \begin{center}\large#1\end{center}
    }
}

\setlength{\droptitle}{-2em}
  \title{}
  \pretitle{\vspace{\droptitle}}
  \posttitle{}
  \author{}
  \preauthor{}\postauthor{}
  \predate{\centering\large\emph}
  \postdate{\par}
  \date{16 March, 2018}

\usepackage{bbm} 
\newcommand{\D}{\text{d}}
\newcommand{\indicator}{\mathbbm{1}}
\newcommand{\hamm}{d_h}
\newcommand{\symhamm}{d_{sh}}
\newcommand{\hz}{\text{Hz}}
\newcommand{\tr}{\text{tr}}
\newcommand{\Iext}{{I_e/A}}
\newcommand\gvn[1][]{\:#1\vert\:}
\usepackage{afterpage}
\usepackage{fancyvrb}
\usepackage{longtable}
\usepackage{booktabs} 
\usepackage{placeins}

\begin{document}

{
\setcounter{tocdepth}{2}
\tableofcontents
}
\subsection{Finding priors for the beta
distribution}\label{finding-priors-for-the-beta-distribution}

Beta distribution is the natural prior for a binomial/bernoulli
distribution. Considering distribution of a binomial variable
\(X\gvn\theta\sim Binom(\theta)\), in order to make its marginalisd
distribution \(P(X) = \int P(X\gvn\theta)P(\theta).d\theta\)
analytically tractable, one of the choice is to assume
\(\theta\sim Beta(M,\alpha)\), so that:

\[
\begin{aligned}
P(\theta) &=\frac{x^{\alpha - 1}(1-x)^{M-\alpha-1}}{B(\alpha,M-\alpha)}
\\
E(\theta) &={\alpha\over M }
\end{aligned}
\]

\subsubsection{Constraint 1:}\label{constraint-1}

\[
\begin{aligned}
P(\theta\le 0.25) = P(\theta\ge 0.75) =  0.05 \\
P(\theta\le 0.75)=0.95
\end{aligned}
\]

Fitted result: \(\theta\sim\) Beta(4.933, 4.932 )

\subsubsection{Constraint 2}\label{constraint-2}

\[
\begin{aligned}
\text{argmax}_\theta(P(\theta))=0.4 \\
P(\theta\le 0.3) = 0.1
\end{aligned}
\]

Fitted result: \(\theta\sim\) Beta(13.863, 20.295 )

\begin{figure}
\centering
\includegraphics{beta_prior_files/figure-latex/constrained-beta-1.pdf}
\caption{\label{fig:constrained-beta}The shape of the beta distribution
constrianed under the respective conditions. Left: Constraint 1. Right:
Constraint 2. (green: PDF, black: CDF)}
\end{figure}

\begin{figure}
\centering
\includegraphics{beta_prior_files/figure-latex/bayes-1.pdf}
\caption{\label{fig:bayes}Inference of posterior distribution on parameter
\(\theta\) given mutatble sequence: 011100101101}
\end{figure}

\subsubsection{\texorpdfstring{Basics for Bayesian inference
\label{sec:bayes-basics}}{Basics for Bayesian inference }}\label{basics-for-bayesian-inference}

\[
\begin{aligned}
\text{likelihood}&: f(\theta) = P(x \gvn \theta) \\
\text{prior}&: f(\theta) = P(\theta) \\
\text{posterior}&: f(\theta)  = P(\theta \gvn x) = \frac{  P(x\gvn \theta) P(\theta)  }{P(x) } \\
\text{marginal likelihood} &: P(x) =  \int P(x\gvn \theta) P(\theta).d\theta
\end{aligned}
\]

The observed toss sequence is: 011100101101

Assume each coin toss is independent from each other, the likelihood of
an observed sequence is only dependent on the total number of heads and
not the sequence it occurred in. Denoting the coin toss as a sequence
\(\{x_i\}\) where \(x_i \in \{0,1\}\), we have

\[
\begin{aligned}
\text{\#head} = \indicator\{x_i=1\} \\
\text{\#head} \sim Binom(|\{x_i\}|,\theta)
\end{aligned}
\] Combining with the prior \(\theta\sim\) Beta(13.863, 20.295 ), I
calculated the marginal likelihood numerically to be \(P(x)=\) 0.11 and
derived the posterior distribution accordingly (see figure
\ref{fig:bayes}). MLE is obtained at \(\hat{\theta}=\) 0.45


\end{document}
